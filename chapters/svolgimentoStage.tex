\chapter{Svolgimento dello \textit{stage}}
\label{cap:descrizione-stage}

\section{Flusso di lavoro}
Descrizione del flusso di lavoro adottato, organizzato in sprint su Jira, con relativa dashboard dei \textit{ticket} da completare
\section{Allineamento metodologico e tecnologico}
Descrizione del primo periodo di studio sulle metodologie aziendale e sulle tecnologie utilizzate
\section{Analisi preventiva dei rischi}
Descrizione dell'analisi dei rischi che potevano sorgere durante il percorso di tirocinio
\section{Creazione delle \textit{user stories}}
Descrizione dell'attività di creazione delle user stories, svolta settimanalmente con il tutor aziendale
\section{Progettazione}
\subsection{Architettura del sistema \textit{Jira}}
Descrizione dell'architettura ad eventi progettata per il sistema Jira
\subsection{Architettura del \textit{Chatbot}}
Descrizione dell'architettura del Chatbot tramite il framework Streamlit.
\section{Codifica}
Descrizione dell'attività di codifica delle componenti \textit{AWS Lambda} e del \textit{chatbot}
\section{Verifica e validazione}
Descrizione del processo di Verifica e validazione delle funzione lambda create, attraverso la creazione di eventi di test.
Per quanto riguarda il chatbot, verrà illustrato il perchè non saranno presenti metriche di test.

\section{Risultato finale}
Descrizione del risultato finale ottenuto, ad alto livello.