\chapter{Svolgimento dello \textit{stage}} 
\label{cap:descrizione-stage}

\section{Flusso di lavoro}
\subsection{Pianificazione delle attività}
In accordo agli obiettivi descritti in \ref{sec:obiettiviStage}, ho redatto, insieme al \textit{tutor} aziendale, una pianficazione settimanale delle attività da svolgere.
Nella tabella \ref{tab:prevAttività} presento la pianificazione di tali attività.

\renewcommand{\arraystretch}{2}
\begin{longtable}{|p{3cm}|p{9cm}|} 
    \hline
    \rowcolor{tableheader}\textbf{Periodo} & \textbf{Descrizione attività} \\
    \hline
    \endfirsthead

    \rowcolor{tableheader}\textbf{Periodo} & \textbf{Descrizione attività} \\
    \hline
    \endhead

    \hline
    \endfoot

    \hline
    \endlastfoot
    \rowcolor{tableevenrow} Prima settimana & \begin{tabular}[t]{@{}p{9cm}@{}}
        - Studio ed apprendimento delle tecnologie di sviluppo \\
        - Definizione delle \textit{user stories} \\
    \end{tabular} \\
    \hline
    \hline
    \rowcolor{tableoddrow} Seconda settimana &  \begin{tabular}[t]{@{}p{9cm}@{}}
        - Creazione \textit{ticket} di \textit{mock} \\
        - Caricamento dei \textit{ticket} all'interno dell'\gls{its} Jira
    \end{tabular} \\
    \hline
    \rowcolor{tableevenrow} Terza settimana & \begin{tabular}[t]{@{}p{9cm}@{}}
        - Reperire i ticket completati da Jira\\
        - Salvataggio \textit{ticket} su \textit{database} MongoDB \\
        - Aggiornamento costante del \textit{database}\\
    \end{tabular} \\
    \hline
    \rowcolor{tableoddrow} Quarta settimana & \begin{tabular}[t]{@{}p{9cm}@{}}
        - Studio \textit{embeddings} per \gls{rag-g}\\
        - Tokenizzazione dei \textit{ticket} contenuti nel \textit{database} MongoDB \\
        - Connessione tra AWS Bedrock e MongoDB per \gls{rag-g}\\
    \end{tabular} \\
    \hline
    \rowcolor{tableevenrow} Quinta settimana & \begin{tabular}[t]{@{}p{9cm}@{}}
        - Creazione di domande di \textit{benchmark}\\
        - Confronto tra i diversi LLM\\
    \end{tabular} \\
    \hline
    \rowcolor{tableoddrow} Sesta settimana & \begin{tabular}[t]{@{}p{9cm}@{}}
        - Aggiornamento del \textit{ticket} Jira con la proposta di risoluzione, generata tramite il sistema di IA Generativa\\
        - Sviluppo di un \textit{chatbot} per l'interrogazione sui \textit{ticket} Jira\\
    \end{tabular} \\ 
    \hline
    \rowcolor{tableevenrow} Settima settimana & \begin{tabular}[t]{@{}p{9cm}@{}}
        - Migliorie al sistema di proposte di risoluzione Jira\\
        - Migliorie al \textit{chatbot}, dando all'utente la possibilità di utilizzare LLM diversi \\
        - Implementare un sistema di autenticazione nel \textit{chatbot} \\
    \end{tabular} \\ 
    \hline
    \rowcolor{tableoddrow} Ottava settimana & \begin{tabular}[t]{@{}p{9cm}@{}}
        - Realizzazione di una presentazione sui progetti sviluppati\\
        - Redazione documento manuale utente \\
        - Redazione documento sull'architettura e sulle scelte progettuali adottate \\
        - Miglioria al \textit{chatbot} per proposte di risoluzione più descrittive \\
    \end{tabular} \\ 
    \hline
    \caption{Pianificazione preventiva da piano di lavoro}
    \label{tab:prevAttività}
\end{longtable}
\section{Analisi preventiva dei rischi}
Descrizione dell'analisi dei rischi che potevano sorgere durante il percorso di tirocinio
\section{Creazione delle \textit{user stories}}
Descrizione dell'attività di creazione delle user stories, svolta settimanalmente con il tutor aziendale
\section{Progettazione}
\subsection{Architettura del sistema \textit{Jira}}
Descrizione dell'architettura ad eventi progettata per il sistema Jira
\subsection{Architettura del \textit{Chatbot}}
Descrizione dell'architettura del Chatbot tramite il framework Streamlit.
\section{Codifica}
Descrizione dell'attività di codifica delle componenti \textit{AWS Lambda} e del \textit{chatbot}
\section{Verifica e validazione}
Descrizione del processo di Verifica e validazione delle funzione lambda create, attraverso la creazione di eventi di test.
Per quanto riguarda il chatbot, verrà illustrato il perchè non saranno presenti metriche di test.

\section{Risultato finale}
Descrizione del risultato finale ottenuto, ad alto livello.