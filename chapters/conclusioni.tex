\chapter{Valutazione retrospettiva}
\label{cap:conclusioni}

\section{Raggiungimento obiettivi}
\subsection{Obiettivi aziendali}
Durante il mio percorso di \textit{stage} sono riuscito a sviluppare le funzionalità richieste dallì'azienda, garantendo che i prodotti rispettassero i requisiti definiti durante l'analisi e che fossero conformi alle aspettative.
Ho raggiunto il 100\% degli obiettivi desritti nelle tabelle \ref{tab:tracciamentoRequisiti} e \ref{tab:obiettiviChatbot}.
Nelle tabelle \ref{tab:resocontoobiettiviJira} e \ref{tab:resocontoobiettiviChatbot}, riporto il resoconto deli obiettivi aziendali dei due progetti:

\subsubsection{Sistema di proposte di risoluzioni}
\renewcommand{\arraystretch}{2}
\begin{longtable}{|p{9cm}|p{2cm}|}
    \hline
    \rowcolor{tableheader}\textbf{Obiettivo} & \textbf{Stato}\\
    \hline
    \endfirsthead

    \rowcolor{tableheader}\textbf{Obiettivo} & \textbf{Stato}\\
    \hline
    \endhead

    \hline
    \endfoot

    \hline
    \endlastfoot

    \rowcolor{tableoddrow} Studio delle tecnologie e studio di fattibilità & Raggiunto \\
    \hline
    \rowcolor{tableevenrow} Creazione dati di \textit{mock} da inserire nell'\gls{its} Jira & Raggiunto \\
    \hline
    \rowcolor{tableoddrow} Reperimento dei ticket su Jira e salvataggio su \textit{database} mongoDB & Raggiunto \\
    \hline
    \rowcolor{tableevenrow} Aggiornamento costante del \textit{database} a nuovi \textit{ticket} risolti su Jira & Raggiunto \\
    \hline
    \rowcolor{tableoddrow} Eseguire \gls{rag} sui \textit{ticket} salvati & Raggiunto \\
    \hline
    \rowcolor{tableevenrow} \textit{Benchmark} su vari modelli per la generazione della risposta & Raggiunto \\
    \hline
    \rowcolor{tableoddrow} Creazione del sistema di proposte di risoluzione su Jira & Raggiunto \\
    \hline
    \caption{Resoconto obiettivi aziendali del sistema di proposte di risoluzione}
    \label{tab:resocontoobiettiviJira}
\end{longtable}

\subsubsection{\textit{Chatbot}}
\begin{longtable}{|p{9cm}|p{2cm}|}
    \hline
    \rowcolor{tableheader}\textbf{Obiettivo} & \textbf{Stato} \\
    \hline
    \endfirsthead

    \rowcolor{tableheader}\textbf{Obiettivo} & \textbf{Stato}\\
    \hline
    \endhead

    \hline
    \endfoot

    \hline
    \endlastfoot

    \rowcolor{tableoddrow} Studio delle tecnologie e studio di fattibilità & Raggiunto \\
    \hline
    \rowcolor{tableevenrow} Implementazione interfaccia \textit{web} per l'interrogazione del \textit{chatbot} & Raggiunto \\
    \hline
    \rowcolor{tableoddrow} Impostazione di un formato di domanda dell'utente da seguire & Raggiunto \\
    \hline
    \caption{Resoconto obiettivi aziendali del \textit{chatbot}}
    \label{tab:resocontoobiettiviChatbot}
\end{longtable}


\subsection{Obiettivi personali}
Gli obiettivi personali che mi sono prefissato prima di iniziare lo \textit{Stage}, descritti nella sezione \ref{sec:obiettiviPersonali}, erano volti a sviluppare competenze tecniche e personali che mi permettessero di crescere dal punto di vista personale e professionale.
Durante il mio percorso di \textit{stage} ho raggiunto tutti gli obiettivi personali che mi ero prefissati, ovvero:
\begin{itemize}
    \item \textbf{Metodologia aziendale}: ho partecipato attivamente ai processi e alle attività aziendali, inerenti allo sviluppo di un prodotto \textit{software}, mettendo in pratica le conoscenze acquisite durante il corso di "Ingegneria del Software". Ho avuto modo di vedere come fosse strutturato il modello di sviluppo \textit{agile} con il \textit{framework} Scum. Ho partecipato alle attività di \textit{stand-up meeting}, svolto irregolarmente per via degli impegni che hanno occupato il mio \textit{tutor} aziendale e \textit{sprint review}. Quest'utlima attività è stata svolta in modo regolare, con cadenza settimanale, con il supporto di strumenti di comunicazione, collaborazione e tracciamento come Slack, Jira e Github. 
    \item \textbf{Conoscenze tecniche}: lo \textit{stage} mi ha permesso di approfondire le mie conoscenze attraverso lo studio di nuove tecnologie e l'applicazione pratica di queste ultime. Lo sviluppo dei due progetti mi ha permesso di lavorare attivamente con \textit{framework} nuovi come Serverless e Streamlit e di approfondire le mie conoscenze con i linguaggi di programmazione come Typescript e Python. Inoltre, ho avuto modo di acquisire nuove conoscenze e competenze con l'utilizzo dei servizi \textit{cloud} offerti da \gls{aws} come \gls{aws} Lambda e \gls{aws} Bedrock.
    \item \textbf{\textit{Problem solving}}: durante il mio percorso di \textit{stage} ho avuto modo di affrontare problemi e sfide che mi hanno permesso di sviluppare le mie capacità di \textit{problem solving}
\end{itemize}

Nella tabella \ref{tab:resocontoobiettiviPersonali} riporto il resoconto degli obiettivi personali raggiunti durante il mio percorso di \textit{stage}:
\renewcommand{\arraystretch}{2}
\begin{longtable}{|p{2cm}|p{6.5cm}|p{1.7cm}|}
    \hline
    \rowcolor{tableheader}\textbf{Codice Obiettivo} & \textbf{Descrizione dell'Obiettivo Personale} & \textbf{Stato} \\
    \hline
    \endfirsthead

    \rowcolor{tableheader}\textbf{Codice Obiettivo} & \textbf{Descrizione dell'Obiettivo Personale} & \textbf{Stato} \\
    \hline
    \endhead

    \hline
    \endfoot

    \hline
    \endlastfoot
    \rowcolor{tableoddrow} OP-1 & Sviluppare competenze con strumenti di comunicazione e collaborazione aziendali come Slack e GitHub & Raggiunto \\
    \hline
    \rowcolor{tableevenrow} OP-2 & Sviluppare competenze con i \textit{framework} utilizzati, come Serverless \textit{framework} e Streamlit & Raggiunto \\
    \hline
    \rowcolor{tableoddrow} OP-3 & Sviluppare competenze con nuovi linguaggi di programmazione come TypeScript e Python. & Raggiunto \\
    \hline
    \rowcolor{tableevenrow} OP-4 & Sviluppare competenze con le tecnologie offerte da \gls{aws}. & Raggiunto \\
    \hline
    \rowcolor{tableoddrow} OP-5 & Partecipare ai processi di sviluppo \textit{software} in ambito aziendale. & Raggiunto \\
    \hline
    \rowcolor{tableevenrow} OP-6 & Comprendere i ritmi e le dinamiche di un lavoro in questo settore. & Raggiunto \\
    \hline
    \caption{Resoconto obiettivi personali}
    \label{tab:resocontoobiettiviPersonali}
\end{longtable}



\section{Difficoltà incontrate}
Durante il mio percorso di \textit{stage} ho incontrato alcune difficiolta che si sono presentate durante le prime settimane di lavoro. Le principali difficoltà incontrate sono qui di seguito descritto con la relativa soluzione adottata:
\begin{itemize}
    \item \textbf{Timore nel chiedere aiuto}: essendo stata la mia prima esperienza lavorativa in un'azienda operante nel settore informatico, ho avuto timore nel chiedere aiuto ai membri più esperti per paura di disturbare il loro lavoro. 
        \begin{itemize} 
        \item \textbf{Soluzione}: ho superato questa difficoltà grazie al clima di collaborazione e supporto che ho trovato in azienda. Ho avuto modo di confrontarmi con il mio \textit{tutor} aziendale e con gli altri membri dell'azienda per chiarire i miei dubbi
        \end{itemize}
    \item \textbf{Difficoltà con l'utilizzo delle \gls{api} di Jira}: durante la seconda e terza settimana di lavoro, come descritto nella tabella \ref{tab:prevAttività}, dovevo creare dei \textit{ticket} di \textit{mock}, inserirli in un progetto Jira e infine reperirli per poterli salvare all'interno del \textit{database}. Ho incontrato delle difficoltà nell'utilizzo delle \gls{api} di Jira adibite all'inserimento e reperimento dei \textit{ticket}. \begin{itemize} \item \textbf{Soluzione}: ho superato questa difficoltà grazie al riferimento alla documentazione dettagliata delle \gls{api} di Jira e al supporto del mio \textit{tutor} aziendale. La strategia di mitigazione del rischio descritto nella sezione \ref{sec:rischi} si è rivelata efficace. \end{itemize}
\end{itemize}

\section{Competenze acquisite}
L'esperienza di \textit{stage} è stata molto importante per la mia crescita professionale e personale. Ho categorizzato le competenze acquisite in tre macroaree:

\begin{itemize}
    \item \textbf{Competenze tecniche}: Durante le settimane di stage, ho utilizzato nuove tecnologie e strumenti di sviluppo come Serverless \textit{framework}, il \textit{framework} Streamlit, linguaggi di programmazione come TypeScript e Python, e i servizi \textit{cloud} offerti da \gls{aws}. Lavorando costantemente su entrambi i progetti con queste tecnologie, ho potuto apprendere e applicare le conoscenze acquisite durante il corso di studi, osservando come vengono implementate in un contesto aziendale.
    \item \textbf{Competenze metodologiche}: L'approccio con nuove tecnologie e strumenti ha richiesto una buona organizzazione, necessaria per apprendere e mettere in pratica le conoscenze rispettando le tempistiche aziendali. Inoltre, ho partecipato attivamente ai processi di sviluppo \textit{software} in un contesto aziendale, il che mi ha permesso di osservare processi ben strutturati e organizzati, comprendendo i ritmi e le dinamiche del lavoro in questo settore.
    \item \textbf{Competenze personali}: Con il passare delle settimane in azienda, ho acquisito maggiore sicurezza nelle mie capacità di comunicazione e di lavoro in team. Svolgere settimanalmente le \textit{sprint review} mi ha permesso di sviluppare la capacità di presentare il lavoro svolto in modo chiaro e conciso, migliorando le mie capacità di comunicazione. Quest'ultima è stata fondamentale quando ho spiegato il lavoro svolto e le tecnologie utilizzate allo stagista che ha preso in mano il progetto dopo di me. Ancora grazie alle \textit{sprint review}, ho sviluppato la capacità di ascolto e di adattamento, imparando a gestire critiche e suggerimenti in modo costruttivo.
\end{itemize}

\section{Divario formativo università e lavoro}
L'esperienza di \textit{stage} si è rivelata un'opportunità per comprendere le relazioni tra il mondo accademico e il mondo del lavoro. Personalmente ritengo che l'università e il mondo professionale siano due mondi complementari, ciascuno con il proprio ruolo e le proprie finalità. L'università fornisce le basi teoriche e le competenze necessarie per affrontare il mondo del lavoro. In particolare, il corso di studi che meglio mi ha preparato per affrontare l'esperienza di \textit{stage} è stato "Ingegneria del Software". Questo corso, grazie anche al progetto didattico svolto, offre una base solida di conoscenze teoriche e pratiche. Il mondo del lavoro, invece, offre l'opportunità di applicare le conoscenze acquisite in un contesto reale, affrontando problemi e sfide che non si presentano durante il percorso di studi.  Questa esperienza ritengo essere fontamentale per consolidare e ampliare le proprie competenze acquisite durante gli studi. \\
L'esperienza di \textit{stage} rappresenta un'opportunità per acquisire nuove competenze. Un esempio significativo è il lavoro in \textit{team}, un aspetto fondamentale nell'ambito dello sviluppo \textit{software}, dove la collaborazione e la comunicazione sono alla base di qualsiasi progetto aziendale. Questo aspetto è stato trattato in modo limitato durante il corso di studi. Il collocamento dello \textit{stage}, ovvero posto in seguito al completamento del percorso di studi e alla terminazione del progetto didattico, ha permesso di arrivare preparati e consapevoli delle metodologie di lavoro, delle abilità comunicative migliorate grazie ai "Diari di bordo" svolti durante il corso di "Ingegneria del Software" e delle competenze tecniche acquisite durante il corso di studi. 
Nonostante quanto ho evidenziato, ritengo che il percorso di studi seguito fornisca una base solida di conoscenze e competenze per affrontare il mondo del lavoro. L'università offre una formazione più ampia e generale, che si concentra sullo sviluppo di competenze trasversali e di pensiero critico, che sono fondamentali per affrontare il mondo del lavoro anche in ambiti diversi. 
\section{Considerazioni finali}
L'esperienza di \textit{stage} presso l'azienda \textit{Zextras} è stata molto formativa e mi ha permesso di crescere molto dal punto di vista professionale e personale. Ho avuto modo di conoscere le metodologie di lavoro in un'azienda operante nel settore informatico, di applicare le conoscenze acquisite durante il corso di studi e di acquisire nuove competenze tecniche e personali. Ritengo che questa esperienza sia anche utile per conoscere tutte le sfaccettature del mondo del lavoro, comprendendo i ritmi e le dinamiche di un'azienda e tutte le persone che vi lavorano grazie alle quali ho passato un'esperienza formativa e molto piacevole.
