\chapter{Valutazione retrospettiva}
\label{cap:conclusioni}

\section{Raggiungimento obiettivi}
\subsection{Obiettivi aziendali}
Durante il mio percorso di \textit{stage} sono riuscito a sviluppare le funzionalità richieste dall'azienda, garantendo che i prodotti rispettassero i requisiti definiti durante l'analisi e che fossero conformi alle loro aspettative.
Ho raggiunto il 100\% degli obiettivi desritti nelle tabelle \ref{tab:tracciamentoRequisiti} e \ref{tab:obiettiviChatbot}. Inoltre ho redatto la documentazione tecnica e il manuale utente per entrambi i progetti, come richiesto dall'azienda.\\
Nelle tabelle \ref{tab:resocontoobiettiviJira} e \ref{tab:resocontoobiettiviChatbot}, riporto il resoconto deli obiettivi aziendali dei due progetti:

\subsubsection{Sistema di proposte di risoluzioni}
\renewcommand{\arraystretch}{2}
\begin{longtable}{|p{9cm}|p{2cm}|}
    \hline
    \rowcolor{tableheader}\textbf{Obiettivo} & \textbf{Stato}\\
    \hline
    \endfirsthead

    \rowcolor{tableheader}\textbf{Obiettivo} & \textbf{Stato}\\
    \hline
    \endhead

    \hline
    \endfoot

    \hline
    \endlastfoot

    \rowcolor{tableoddrow} Studio delle tecnologie e studio di fattibilità & Raggiunto \\
    \hline
    \rowcolor{tableevenrow} Creazione dati di \textit{mock} da inserire nell'\gls{its} Jira & Raggiunto \\
    \hline
    \rowcolor{tableoddrow} Reperimento dei ticket su Jira e salvataggio su \textit{database} mongoDB & Raggiunto \\
    \hline
    \rowcolor{tableevenrow} Aggiornamento costante del \textit{database} a nuovi \textit{ticket} risolti su Jira & Raggiunto \\
    \hline
    \rowcolor{tableoddrow} Eseguire \gls{rag} sui \textit{ticket} salvati & Raggiunto \\
    \hline
    \rowcolor{tableevenrow} \textit{Benchmark} su vari modelli per la generazione della risposta & Raggiunto \\
    \hline
    \rowcolor{tableoddrow} Creazione del sistema di proposte di risoluzione su Jira & Raggiunto \\
    \hline
    \caption{Resoconto obiettivi aziendali del sistema di proposte di risoluzione}
    \label{tab:resocontoobiettiviJira}
\end{longtable}

\subsubsection{\textit{Chatbot}}
\renewcommand{\arraystretch}{2}
\begin{longtable}{|p{9cm}|p{2cm}|}
    \hline
    \rowcolor{tableheader}\textbf{Obiettivo} & \textbf{Stato} \\
    \hline
    \endfirsthead

    \rowcolor{tableheader}\textbf{Obiettivo} & \textbf{Stato}\\
    \hline
    \endhead

    \hline
    \endfoot

    \hline
    \endlastfoot

    \rowcolor{tableoddrow} Studio delle tecnologie e studio di fattibilità & Raggiunto \\
    \hline
    \rowcolor{tableevenrow} Implementazione interfaccia \textit{web} per l'interrogazione del \textit{chatbot} & Raggiunto \\
    \hline
    \rowcolor{tableoddrow} Impostazione di un formato di domanda dell'utente da seguire & Raggiunto \\
    \hline
    \caption{Resoconto obiettivi aziendali del \textit{chatbot}}
    \label{tab:resocontoobiettiviChatbot}
\end{longtable}


\subsection{Obiettivi personali}
Gli obiettivi personali che mi sono prefissato prima di iniziare lo \textit{stage}, descritti nella sezione \ref{sec:obiettiviPersonali}, erano volti a sviluppare competenze tecniche e personali che mi permettessero di crescere dal punto di vista personale e professionale.
Durante il mio percorso di \textit{stage} ho raggiunto tutti gli obiettivi personali che mi ero prefissato.
\begin{itemize}
    \item \textbf{Metodologia aziendale}: ho partecipato attivamente ai processi e alle attività aziendali, inerenti allo sviluppo di un prodotto \textit{software}, mettendo in pratica le conoscenze acquisite durante il corso di "Ingegneria del Software". Ho avuto modo di vedere come fosse strutturato il modello di sviluppo \textit{agile} con il \textit{framework} Scum. Ho partecipato alle attività di \textit{stand-up meeting}, in numero limitato a causa  dei numerosi impegni che il mio \textit{tutor} aziendale aveva e \textit{sprint review}. Quest'ultima attività è stata svolta in modo regolare, con cadenza settimanale, con il supporto di strumenti di comunicazione, collaborazione e tracciamento come Slack, Jira e Github. 
    \item \textbf{Conoscenze tecniche}: lo \textit{stage} mi ha permesso di approfondire le mie conoscenze attraverso lo studio di nuove tecnologie e l'applicazione pratica di queste ultime. Lo sviluppo dei due progetti mi ha permesso di lavorare attivamente con \textit{framework} nuovi come Serverless e Streamlit e di approfondire le mie conoscenze con i linguaggi di programmazione come Typescript e Python. Inoltre, ho avuto modo di acquisire nuove conoscenze e competenze con l'utilizzo dei servizi \textit{cloud} offerti da \gls{aws} come \gls{aws} Lambda e \gls{aws} Bedrock.
    \item \textbf{\textit{Problem solving}}: durante il mio percorso di \textit{stage} ho avuto modo di affrontare problemi e sfide che mi hanno permesso di migliorare le mie capacità di \textit{problem solving}
\end{itemize}

Nella tabella \ref{tab:resocontoobiettiviPersonali} riporto il resoconto degli obiettivi personali raggiunti durante il mio percorso di \textit{stage}:
\renewcommand{\arraystretch}{2}
\begin{longtable}{|p{2cm}|p{6.5cm}|p{1.7cm}|}
    \hline
    \rowcolor{tableheader}\textbf{Codice Obiettivo} & \textbf{Descrizione dell'Obiettivo Personale} & \textbf{Stato} \\
    \hline
    \endfirsthead

    \rowcolor{tableheader}\textbf{Codice Obiettivo} & \textbf{Descrizione dell'Obiettivo Personale} & \textbf{Stato} \\
    \hline
    \endhead

    \hline
    \endfoot

    \hline
    \endlastfoot
    \rowcolor{tableoddrow} OP-1 & Sviluppare competenze con strumenti di comunicazione e collaborazione aziendali come Slack e GitHub & Raggiunto \\
    \hline
    \rowcolor{tableevenrow} OP-2 & Sviluppare competenze con i \textit{framework} utilizzati, come Serverless \textit{framework} e Streamlit & Raggiunto \\
    \hline
    \rowcolor{tableoddrow} OP-3 & Sviluppare competenze con nuovi linguaggi di programmazione come TypeScript e Python. & Raggiunto \\
    \hline
    \rowcolor{tableevenrow} OP-4 & Sviluppare competenze con le tecnologie offerte da \gls{aws}. & Raggiunto \\
    \hline
    \rowcolor{tableoddrow} OP-5 & Partecipare ai processi di sviluppo \textit{software} in ambito aziendale. & Raggiunto \\
    \hline
    \rowcolor{tableevenrow} OP-6 & Comprendere i ritmi e le dinamiche di un lavoro in questo settore. & Raggiunto \\
    \hline
    \caption{Resoconto obiettivi personali}
    \label{tab:resocontoobiettiviPersonali}
\end{longtable}



\section{Difficoltà incontrate}
Durante il mio percorso di \textit{stage} ho incontrato alcune difficoltà che si sono presentate durante le prime settimane di lavoro. Le principali difficoltà incontrate sono qui di seguito descritto con le relative soluzioni adottate:
\begin{itemize}
    \item \textbf{Timore nel chiedere aiuto}: essendo stata la mia prima esperienza lavorativa in un'azienda operante nel settore informatico, ho avuto timore nel chiedere aiuto ai membri più esperti, per paura di disturbare il loro lavoro. 
        \begin{itemize} 
        \item \textbf{Soluzione}: ho superato questa difficoltà grazie al clima di collaborazione presente in azienda, che mi ha permesso di liberarmi di ogni timore. Ho avuto modo di confrontarmi con il mio \textit{tutor} aziendale e con gli altri membri dell'azienda per chiarire i miei dubbi
        \end{itemize}
    \item \textbf{Difficoltà con l'utilizzo delle \gls{api} di Jira}: durante la seconda e terza settimana di lavoro, come descritto nella tabella \ref{tab:prevAttività}, dovevo creare dei \textit{ticket} di \textit{mock}, inserirli in un progetto Jira e infine reperirli per poterli salvare all'interno del \textit{database}. Ho incontrato delle difficoltà nell'utilizzo delle \gls{api} di Jira adibite all'inserimento e reperimento dei \textit{ticket}. \begin{itemize} \item \textbf{Soluzione}: ho superato questa difficoltà grazie al riferimento alla documentazione dettagliata delle \gls{api} di Jira e al supporto del mio \textit{tutor} aziendale. La strategia di mitigazione del rischio descritto nella sezione \ref{sec:rischi} si è rivelata efficace. \end{itemize}
\end{itemize}

\section{Competenze acquisite}
L'esperienza di \textit{stage} è stata molto importante per la mia crescita professionale e personale. Ho categorizzato le competenze acquisite in tre macroaree:

\begin{itemize}
    \item \textbf{Competenze tecniche}: Durante le settimane di stage, ho utilizzato nuove tecnologie e strumenti di sviluppo come Serverless \textit{framework}, il \textit{framework} Streamlit, linguaggi di programmazione come TypeScript e Python, e i servizi \textit{cloud} offerti da \gls{aws}. Lavorando costantemente su entrambi i progetti con queste tecnologie, ho potuto apprendere e applicare le conoscenze acquisite durante il corso di studi, osservando come vengono implementate in un contesto aziendale.
    \item \textbf{Competenze metodologiche}: L'approccio a nuove tecnologie e strumenti ha richiesto una solida organizzazione, fondamentale per apprendere e applicare le conoscenze, rispettando le tempistiche aziendali. Inoltre, ho partecipato attivamente ai processi di sviluppo \textit{software} in un contesto aziendale, il che mi ha permesso di osservare processi ben strutturati e organizzati, comprendendo appieno i ritmi e le dinamiche del lavoro in questo settore.
    \item \textbf{Competenze personali}: Con il passare delle settimane in azienda, ho acquisito una maggiore sicurezza nelle mie capacità di comunicazione e di lavoro in \textit{team}. Partecipare settimanalmente alle \textit{sprint review} mi ha permesso di sviluppare la capacità di presentare il lavoro svolto in modo chiaro e conciso, affinando le mie capacità di comunicazione. Quest’ultima è stata ulteriormente rafforzata grazie alla presentazione che ho realizzato e presentato all’azienda, la quale mi ha permesso di affinare ulteriormente le mie abilità comunicative. Spiegare il lavoro svolto e le tecnologie utilizzate allo stagista che ha preso in mano il progetto dopo di me è stato un altro momento chiave per mettere in pratica questa abilità. Ancora grazie alle \textit{sprint review}, ho sviluppato una maggiore capacità di ascolto e di adattamento, imparando a gestire critiche e suggerimenti in modo costruttivo. In conclusione, ho migliorato anche il mio approccio al \textit{problem solving}, affrontando e risolvendo in modo efficace le sfide che si sono presentate durante il mio percorso di \textit{stage}.
\end{itemize}

\section{Divario formativo università e lavoro}
L'esperienza \textit{stage} mi ha aiutato a cogliere le connessioni tra il mondo accademico e il mondo professionale. Personalmente, ritengo che l'università e il mondo del lavoro siano due realtà complementari, ciascuna con il proprio ruolo e i propri obiettivi. L'università fornisce le basi teoriche e le competenze necessarie per affrontare il mondo del lavoro. In particolare, il corso di "Ingegneria del Software" ha avuto un ruolo fondamentale nella mia preparazione allo \textit{stage}, grazie alle solide conoscenze teorico-pratiche acquisite, arricchite dal progetto didattico.
Il mondo del lavoro, invece, consente di applicare le conoscenze apprese in un contesto reale, affrontando problemi e sfide che difficilmente emergono durante gli studi. Ritengo che questa esperienza sia fondamentale per rafforzare e ampliare le competenze che si sviluppano durante il percorso universitario.
Lo \textit{stage}, inoltre, è un’opportunità per migliorare abilità essenziali come il lavoro in \textit{team}, cruciale nello sviluppo di applicativi \textit{software}, dove collaborazione e comunicazione sono alla base di qualsiasi progetto aziendale. Questo aspetto è stato trattato in modo limitato durante il corso di studi. \\ 
Il collocamento dello \textit{stage}, avvenuto dopo il completamento del percorso accademico e del progetto didattico, mi ha permesso di affrontare questa esperienza con maggiore consapevolezza delle metodologie di lavoro, con abilità comunicative rafforzate grazie ai “Diari di bordo” del corso di “Ingegneria del Software”, e competenze tecniche acquisite durante gli studi.
Considerato quanto ho descritto, ritengo che il percorso di studi offra una solida base di conoscenze e competenze per affrontare il mondo del lavoro. In aggiunta, ritengo che l'università offre una formazione ampia e generale, focalizzata sullo sviluppo delle competenze trasversali e del pensiero critico, essenziali per orientarsi nel mondo professionale anche in settori diversi.
\section{Considerazioni finali}
L'esperienza di \textit{stage} presso Zero12 ritengo essere stata estremamente positiva e formativa. Mi ha consentito di sviluppare e migliorare le mie competenze tecniche, metodologiche e personali, applicandole concretamente in un contesto aziendale. Ho avuto modo di lavorare su due progetti molto interessanti, che mi hanno permesso di approfondire le mie conoscenze sui servizi \textit{cloud} offerti da \gls{aws}, sui \textit{framework} Serverless e Streamlit e sui linguaggi di programmazione TypeScript e Python. Inoltre, ho potuto esplorare da vicino il tema della \gls{iaGenerativag}, che mi ha sempre affascinato, e che ho potuto approfondire grazie ai due progetti. L'ambiente di lavoro in cui sono state inserito è stato altamente stimolante, grazie alla presenza di membri esperti e competenti, che mi hanno supportato e con i quali ho potuto scambiare discutere sulle tematiche che ho affrontato durante il mio percorso di \textit{stage}.
In conclusione, sono molto soddisfatto dell'esperienza di \textit{stage} svolta presso Zero12, che mi ha permesso di crescere dal punto di vista professionale e personale, preparandomi al meglio per le sfide future nel mondo del lavoro.