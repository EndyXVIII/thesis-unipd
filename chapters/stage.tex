
\chapter{Lo \textit{stage}}
\label{cap:lo-stage}



\section{Il ruolo degli stage nell'azienda}
Zero12 crede fortemente nella collaborazione tramite \textit{stage} con gli studenti dell'Università di Padova. Grazie all'evento \gls{stageItg}, un' evento promosso da Confindustria Veneto Est e l'Universita degli studi di Padova, l'azienda permette agli studenti di svolgere progetti di \textit{stage} presso la loro sede.
Gli stage per l'azienda, rappresentano una doppia opportunità. Da una parte, hanno modo di inserire nuove figure professionali nel proprio organico, dall’altra gli studenti hanno la possibilità di inserirsi nel mondo del lavoro e poter applicare le proprie conoscenze acquisite durante il percorso di studi.
Durante le otto settimane di \textit{stage} ho avuto l’occasione di conoscere quasi tutti i membri dell’azienda, comprendere il loro metodo di lavoro e mettere in pratica le mie conoscenze. Il mio percorso è stato guidato dal \textit{tutor} aziendale che mi è stato assegnato, che coordinava le attività da svolgere e che mi forniva supporto in caso di dubbi o problemi. 
L'accoglienza e l'ambiente di lavoro sono molto positivi, dimostrando una grande attenzione nei confronti degli stagisti. 
Questa attenzione, è testimoniata anche dalla presenza di molti dipendenti che attualmente lavorano in azienda dopo aver svolto uno \textit{stage} in Zero12.
I progetti assegnati durante gli \textit{stage} vengono concepiti in base a delle reali esigenze interne all’azienda. Essi sono mirati a migliorare la comunicazione tra i collaboratori, automatizzare delle attività manuali e a migliorare la qualità del lavoro svolto.

\section{Introduzione al progetto}
Durante il mio \textit{stage} ho avuto modo di lavorare non solo al progetto inizialmente proposto, ma anche allo sviluppo di un progetto aggiuntivo, strettamene correlato al primo.
Di seguito descrivo il progetto inizialmente proposto dall'azienda all'evento \gls{stageItg} e il progetto aggiuntivo concepito e sviluppato durante lo \textit{stage}.
\subsection{Sistema di proposte di risoluzioni a \textit{ticket} Jira} \label{sec:spiegazioneJira}
L'idea del progetto nasce come strumento che l'azienda può utilizzare per velocizzare il supporto tecnico, che però può essere estesa ai clienti dell'azienda stessa. Nel contesto del progetto, viene immaginata un'azienda che vende componenti \textit{hardware}. 
Nel caso di problematiche con la componente comprata, il cliente chiama il servizio clienti dell'azienda, e l'operatore inserisce un \textit{ticket} all'interno dell'\gls{its} aziendale, che nel contesto del progetto è Jira. Immaginando questa iterazione tra cliente e addetto al servizio clienti per molte volte nel corso degli anni, i \textit{ticket} inseriti all'interno del sistema saranno numerosi e molto simili tra loro. 
Questo implica che in caso di nuovi \textit{ticket} inseriti, è molto probabile che ci siano \textit{ticket} simili già risolti in passato. 
Ed è qui che da Zero12 nasce l'idea di creare un sistema di proposte di risoluzioni automatiche per i \textit{ticket} inseriti all'interno di un \gls{its}, che permetta di velocizzare il processo di risoluzione del problema. Le proposte di risoluzione vengono generate grazie all'utilizzo della IA Generativa, in base al contesto dei \textit{ticket} risolti in precedenza.

\begin{figure}[H]
    \centering
    \includegraphics[width=0.85\textwidth]{ideaJira.png}
    \caption{Schema che rappresenta l'idea del progetto principale}
    \small \textbf{Fonte:} \href{https://www.flaticon.com/free-icon/company_4812244}{Icona azienda: https://www.flaticon.com} \href{https://www.flaticon.com/free-icon/box_4601560}{Icona prodotto: https://www.flaticon.com} \href{https://www.flaticon.com/free-icon/assistant_1442194}{Icona assistente: https://www.flaticon.com} \href{https://www.creativefabrica.com/it/product/ai-brain-outline-icon/} {Icona IA Generativa: https://www.creativefabrica.com}

    \label{fig:ideaJira}
\end{figure}
\subsection{\textit{Chatbot} per proposte di risoluzioni}
Durante lo sviluppo del progetto descritto nel paragrafo precedente, da parte del mio \textit{tutor} aziendale, è nata l'idea di creare un ulteriore strumento che permetta di svolgere la stessa funzionalità del sistema di proposte di risoluzioni, ma in modo differente. Lo strumento aggiuntivo è un \textit{chatbot} che permette di generare proposte di risoluzione, in base al testo di un \textit{ticket} inserito all'interno della chat. 
Sempre grazie all'utilizzo della IA Generativa, il \textit{chatbot} genera proposte di risoluzione in base al contesto dei \textit{ticket} risolti in precedenza. A differenza del sistema Jira, il \textit{chatbot} propone le risoluzioni in modo più descrittivo e dettagliato, e permette di interagire con l'assistente virtuale, per ottenere informazioni più dettagliate sulle proposte di risoluzione.
\begin{figure}[H]
    \centering
    \includegraphics[width=0.85\textwidth]{ideaChatbot.png}
    \caption{Schema che rappresenta l'idea del progetto aggiuntivo}
    \label{fig:ideaChatbot}
    \small \textbf{Fonte:} \href{https://uxwing.com/chatbot-icon/}{Icona chatbot: https://uxwing.com} \href{https://www.iconfinder.com/icons/351012/field_input_search_icon}{Icona \textit{input} : https://www.iconfinder.com} \href{https://www.flaticon.com/free-icon/text-file_5116156} {Icona testo: https://www.flaticon.com}

\end{figure}
\section{Importanza del progetto}
Come descritto nel paragrafo precedente, il progetto ideato da Zero12 di generare proposte di risoluzioni automatiche per i nuovi \textit{ticket} inseriti all'interno di un \gls{its} rappresenterebbe un grande vantaggio per l'azienda. Innanzitutto, 
grazie allo sviluppo di questo progetto, l'azienda dispone di dati sufficienti per poter implementare un sistema di proposte di risoluzioni automatiche, utilizzando i servizi di cui l'azienda è \textit{partner}. In secondo luogo, permette di avere una base solida nel 
caso in cui un cliente desideri implementare un sistema simile a quello sviluppato. Questo permetterebbe di presentare già un prototipo funzionante, che dimostra tutte le potenzialità del sistema, e che può essere facilmente adattato alle specifiche del cliente.
Un altro punto importante del progetto è l'utilizzo di IA Generativa, che sta diventando sempre più utilizzato in ambito aziendale, data la sua capacità di generare risposte in base al contesto in cui viene inserita.
Per quanto riguarda il progetto aggiuntivo, la creazione di un \textit{chatbot} per le proposte di risoluzione riveste un'importanza ancora maggiore. Questo perchè il \textit{chatbot} permette di interagire con l'assistente virtuale e di ottenere informazioni più dettagliate sulle proposte di risoluzione.
\section{Obiettivi dello \textit{stage}}
Durante i primi giorni di \textit{stage} ho definito assieme al mio \textit{tutor} aziendale gli obiettivi del progetto.
Di seguito nella tabella elenco gli obiettivi da raggiungere:
\renewcommand{\arraystretch}{2}
\begin{longtable}{|p{10cm}|p{2cm}|}
    \hline
    \rowcolor{tableheader}\textbf{Obiettivo} & \textbf{Importanza} \\
    \hline
    \endfirsthead

    \rowcolor{tableheader}\textbf{Obiettivo} & \textbf{Importanza} \\
    \hline
    \endhead

    \hline
    \endfoot

    \hline
    \endlastfoot
    \rowcolor{tableoddrow} Studio delle tecnologie e studio di fattibilità & Obbligatorio \\
    \hline
    \rowcolor{tableevenrow} Creazione dati di \textit{mock} da inserire nell' \gls{its} Jira & Obbligatorio \\
    \hline
    \rowcolor{tableoddrow} Reperimento dei ticket su Jira e salvataggio su \textit{database} mongoDB & Obbligatorio \\
    \hline
    \rowcolor{tableoddrow} Aggiornamento costante del \textit{database} a nuovi \textit{ticket} risolti su Jira & Obbligatorio \\
    \hline
    \rowcolor{tableoddrow} Eseguire \gls{rag}, una tecnica avanzata di elaborazione di linguaggio naturale, sui \textit{ticket} salvati & Obbligatorio \\
    \hline
    \rowcolor{tableoddrow} \textit{Benchmark} su vari modelli per la generazione della risposta & Obbligatorio \\
    \hline
    \rowcolor{tableoddrow} Creazione del sistema di proposte di risoluzione su Jira & Obbligatorio \\
    \hline
    \caption{Obiettivi da raggiungere per il sistema di proposte di risoluzione su Jira}
    \label{tab:obiettiviJira}
\end{longtable}
\noindent
Durante lo sviluppo del progetto, grazie il buon andamento che ho avuto, al mio \textit{tutor} aziendale è nata l'idea di sviluppare il progetto aggiuntivo relativo al \textit{chatbot} per le proposte di risoluzione.
Di seguito elenco gli obiettivi extra da raggiungere:
\begin{longtable}{|p{10cm}|p{2cm}|}
    \hline
    \rowcolor{tableheader}\textbf{Obiettivo} & \textbf{Importanza} \\
    \hline
    \endfirsthead

    \rowcolor{tableheader}\textbf{Obiettivo} & \textbf{Importanza} \\
    \hline
    \endhead

    \hline
    \endfoot

    \hline
    \endlastfoot
    \rowcolor{tableoddrow} Studio delle tecnologie e studio di fattibilità & Desiderabile \\
    \hline
    \rowcolor{tableevenrow} Implementazione interfaccia \textit{web} per l'interrogazione del \textit{chatbot} & Desiderabile \\
    \hline
    \rowcolor{tableevenrow} Impostazione di un formato di domanda dell'utente da seguire & Desiderabile \\
    \hline
    \caption{Obiettivi da raggiungere del progetto \textit{chatbot}}
    \label{tab:obiettiviChatbot}
\end{longtable}
\section{Vincoli dello \textit{stage}}
\subsection{Vincoli tecnologici}
I vincoli tecnologici imposti dall'azienda includevano l'utilizzo delle tecnologie utilizzate nello \textit{stack} aziendale, descritte nella sezione \ref{sec:tecnologie}. In aggiunta, 
mi è stato chiesto di utilzzare le \gls{api} che fornisce Jira, per il caricamento di \textit{ticket} di \textit{mock} all'interno di un progetto Jira, e per il recupero dei \textit{ticket} risolti da inserire nel \textit{database}.
Mi è stato inoltre chiesto di utilizzare i \textit{webhook} di Jira, per impostare un \textit{trigger} che invii eventi al sistema che ho sviluppato, in modo che alla chiusura di un \textit{ticket} su Jira, il sistema aggiorna automaticamente il \textit{database} con il nuovo \textit{ticket} completato.
\subsection{Vincoli di rendicontazione}
Durante il mio periodo di \textit{stage}, l'azienda mi ha richiesto la stesura di una tabella riassuntiva dei risultati del \textit{benchmark} dei vari modelli testati. I risultati dei vari modelli sono stati assegnati secondo metriche di valutazione da me impostate e motivate.
Mi è stato richiesta la stesura di un documento in cui spiego le scelte adottate durante lo sviluppo del progetto, indicando le componenti sviluppate, le librerie utilizzate e le motivazioni dietro le scelte fatte.
Mi è stata richiesta anche la stesura di un manuale utente, in cui spiego come configurare correttamente il sistema su Jira, e come utilizzarlo. Quest'ultimo documento comprende anche una sezione in cui descrivo il corretto utilizzo del \textit{chatbot}.\\
Durante l'ultima settimana infine, mi è stato richiesto di preparare una presentazione sui progetti sviluppati, che avrei dovuto esporre a tutti i membri presenti dell'azienda. Questa presentazione doveva contenere le idee dei progetti, la loro importanza, i risultati ottenuti e le possibile evoluzioni future.
\subsection{Vincoli temporali}
Il progetto di \textit{stage} è stato ideato per essere sviluppato in otto settimane, limite massimo di tempo per lo svolgimento del tirocinio curricolare. Nello specifico, è stato programmato in giornate lavorative da 8 ore ciascuna, per un totale di non più di 320 ore, per un totale di 40 ore settimanali.

\section{Scelta dello \textit{stage}} \label{sec:sceltaStage}
Durante l'evento \gls{stageItg} ho avuto modo di connoscere numerose azienda, e di ascoltare i loro progetti di \textit{stage}. 
La scelta del progetto di Zero12 è stata basata su una serie di fattori che descrivo qui di seguito:
\begin{itemize}
    \item \textbf{Tema del progetto}: Il progetto sul supporto tecnico facilitato con l'aiuto della IA Generativa, come descritto nel paragrafo \ref{sec:spiegazioneJira}, mi ha subito colpito, in quanto mi avrebbe permesso di approfondire l'utilizzo di questa tecnologia, sempre più diffusa in vari settori. Ho scelto questo progetto anche per la sua utilità in ambito aziendale e la possibilità di poterlo estendere a clienti reali.
    \item \textbf{Prima impressione}: Come ho descritto all'inizio del paragrafo, ho avuto modo l'opportunità di parlare con diverse aziende. Il colloquio che ho avuto con Zero12 è stato quello che mi è rimasto più impresso, in quanto si sono dimostrati molto disponibili nel rispondere alle mie domande, molto chiari nelle loro descrizioni e molto accoglienti. Dopo il colloquio, e a seguito dell'ottima impressione che ho ricevuto, mi sono interessato maggiormente al progetto da loro proposto.
    \item \textbf{Tecnologie utilizzate}: I giorni prima l'evento, ho avuto modo di esaminare le tecnologie richieste per lo sviluppo del progetto. Tra queste, vi erano tecnologie a me già familiari, come Typescript, Node.js e MongoDB. Il restante delle tecnologie, erano legate ai servizi \textit{cloud} offerte da \gls{aws}, che ho avuto modo di accennare durante il mio percorso scolastico. Ho scelto questo progetto anche per la possibilità di approfondire queste tecnologie e di poterle utilizzare in ambito aziendale.
\end{itemize}

\section{Obiettivi personali}
Prima dell'inizio dello \textit{stage}, mi sono prefissato degli obiettivi che volevo raggiungere grazie a questa esperienza, qui di seguito elencati:
\begin{itemize}
    \item \textbf{Approfondimento e apprendimento di nuove tecnologie}: Come ho accennato nel paragrafo precedente, ero già familiare con alcune tecnologie e dei servizi \textit{cloud} richiesti per lo sviluppo del progetto. Tuttavia, non ho mai l'opportunità di esplorarle a fondo e sfruttarle al loro massimo potenziale. Grazie a queste esperienza, desideravo approfondire al meglio queste tecnologie e utilizzarle in ambito aziendale.
    \item \textbf{Introduzione al mondo del lavoro}: Questa è stata la mia prima esperienza lavorativa in un'azienda collegata al mio percorso di studi. Ho avuto modo di conoscere alcune realtà aziendali, ma l'esperienza in Zero12 è stata la prima ad offrirmi un'opportunità lavorativa in questo specifico settore. Ho voluto cogliere questa occasione per comprendere come si lavori in questo tipo di aziende, le loro metodologie e la loro organizzazione interna. 
    \item \textbf{Equilibrio vita-lavoro}: Essendo la mia esperienza lavorativa più lunga ed organizzata in giornate da 8 ore, ho voluto sfruttare questa occasione per imparare a gestire il mio tempo in modo più efficiente. L'obiettivo era quello di trovare un equilibrio tra il lavoro e vita privata.
\end{itemize}