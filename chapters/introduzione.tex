
\chapter{Contesto lavorativo}
\label{cap:introduzione}
\section{Profilo aziendale}
Zero12 s.r.l. è una \textit{startup} italiana fondata nel 2012 con attualmente due sedi operative, la prima a Padova (dove ho svolto il periodo di \textit{stage}) e la seconda ad Empoli. 
L'azienda è \textit{partner} \gls{aws} ed è specializzata nella realizzazione di applicazioni \textit{cloud native} e dello sviluppo di applicazioni e servizi \textit{web} e \textit{mobile}, offrendo ai propri clienti, proveniente da ambiti diversificati, soluzioni flessibili e scalibili.\\
Il team di Zero12 s.r.l. non si limita alla realizzazione di applicazioni, ma si estende alla creazione di un percorso di innovazione condiviso con i clienti. Questo percorso mira a integrare nelle realtà aziendali le tecnologie più adatte alle specifiche esigenze, assicurando un miglioramento continuo dei processi aziendali e un’efficace presenza sul web.\\
L'azienda è parte di Var Group s.p.a, un' azienda italiana che supporta i propri \textit{partner} con servizi e competenze per raggiungere gli obiettivi.
\section{Organizzazione aziendale}
Durante il mio periodo di \textit{stage} ho avuto modo osservare in prima persona l'organizzazione in \textit{team} di lavoro che l'azienda adotta.
I ruoli che ho potuto osservare sono rappresentati di seguito:
\begin{table}[ht]
    \centering
    \begin{tabular}{|>{\raggedright\arraybackslash}p{3cm}|>{\raggedright\arraybackslash}p{8.5cm}|}
    \hline
    \textbf{Ruolo} & \textbf{Descrizione} \\
    \hline
    \textbf{\textit{CEO}} & Il \gls{ceog} rappresenta il ruolo più alto all'interno dell'azienda. È responsabile di definire la strategia e le decisioni più importanti dell'azienda. Supervisiona i processi aziendali e gestisce i contatti con potenziali nuovi clienti. \\
    \hline
    \textbf{\textit{Project Manager}} & Il \textit{project manager} è la figura che si occupa di coordinare il proprio team di lavoro, gestendo le risorse e le tempistiche di progetto e interfacciandosi con il cliente. Il mio tutor aziendale, svolge anche il ruolo di \textit{project manager} e ho avuto modo di osservare il suo lavoro in prima persona.\\
    \hline
    Ruolo 3 & Descrizione del ruolo 3. \\
    \hline
    \end{tabular}
    \caption{Ruoli nell'organizzazione aziendale}
    \label{tab:organizzazione}
\end{table}

\section{Processi e tecnologie}
\subsection{Processi}
Descrizione dei principali processi di sviluppo dell'azienda.
\subsection{Tecnologie utilizzate}
Elenco e breve spiegazione delle tecnologie utilizzate all'interno dell'azienda
Esempio:
\subsubsection{AWS}
\subsubsection{Typescript}
\subsubsection{Serverless Framework}

%\noindent Esempio di utilizzo di un termine nel glossario \\
%\gls{api}. \\

%\noindent Esempio di citazione in linea \\
%\cite{site:agile-manifesto}. \\

%\noindent Esempio di citazione nel pie' di pagina \\
%citazione\footcite{womak:lean-thinking} \\

\section{Tipologia di clientela}
Analisi della tipologia di clientela dell'azienda.

\section{Propensione all'innovazione}
Valutazione dell'approccio dell'azienda verso l'innovazione e nuove tecnologie

%\section{Organizzazione del testo}

%\begin{description}
    %\item[{\hyperref[cap:introduzione]{Il primo capitolo}}] descrive ...
    %\item[{\hyperref[cap:lo-stage]{Il secondo capitolo}}] descrive ...
    
    %\item[{\hyperref[cap:descrizione-stage]{Il terzo capitolo}}] approfondisce ...
    
    %\item[{\hyperref[cap:conclusioni]{Il quarto capitolo}}] approfondisce ...
    
%\end{description}

%Riguardo la stesura del testo, relativamente al documento sono state adottate le seguenti convenzioni tipografiche:
%\begin{itemize}
	%\item gli acronimi, le abbreviazioni e i termini ambigui o di uso non comune menzionati vengono definiti nel glossario, situato alla fine del presente documento;
	%\item per la prima occorrenza dei termini riportati nel glossario viene utilizzata la seguente nomenclatura: \emph{parola}\glsfirstoccur;
	%\item i termini in lingua straniera o facenti parti del gergo tecnico sono evidenziati con il carattere \emph{corsivo}.
%\end{itemize}
