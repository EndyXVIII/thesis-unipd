
\chapter{Contesto lavorativo}
\label{cap:introduzione}
\section{Profilo aziendale}
Descrizione breve dell'azienda,

\section{Dominio applicativo}
Panoramica del settore in cui opera l'azienda.

\section{Organizzazione aziendale}
Struttura interna e ruoli chiave nell'azienda,

\section{Processi e tecnologie}
\subsection{Processi}
Descrizione dei principali processi di sviluppo dell'azienda.
\subsection{Tecnologie utilizzate}
Elenco e breve spiegazione delle tecnologie utilizzate all'interno dell'azienda
Esempio:
\subsubsection{AWS}
\subsubsection{Typescript}
\subsubsection{Serverless Framework}

\noindent Esempio di utilizzo di un termine nel glossario \\
\gls{api}. \\

\noindent Esempio di citazione in linea \\
\cite{site:agile-manifesto}. \\

\noindent Esempio di citazione nel pie' di pagina \\
citazione\footcite{womak:lean-thinking} \\

\section{Tipologia di clientela}
Analisi della tipologia di clientela dell'azienda.

\section{Propensione all'innovazione}
Valutazione dell'approccio dell'azienda verso l'innovazione e nuove tecnologie

\section{Organizzazione del testo}

\begin{description}
    \item[{\hyperref[cap:lo-stage]{Il primo capitolo}}] descrive ...
    \item[{\hyperref[cap:processi-metodologie]{Il secondo capitolo}}] descrive ...
    
    \item[{\hyperref[cap:descrizione-stage]{Il terzo capitolo}}] approfondisce ...
    
    \item[{\hyperref[cap:analisi-requisiti]{Il quarto capitolo}}] approfondisce ...
    
\end{description}

Riguardo la stesura del testo, relativamente al documento sono state adottate le seguenti convenzioni tipografiche:
\begin{itemize}
	\item gli acronimi, le abbreviazioni e i termini ambigui o di uso non comune menzionati vengono definiti nel glossario, situato alla fine del presente documento;
	\item per la prima occorrenza dei termini riportati nel glossario viene utilizzata la seguente nomenclatura: \emph{parola}\glsfirstoccur;
	\item i termini in lingua straniera o facenti parti del gergo tecnico sono evidenziati con il carattere \emph{corsivo}.
\end{itemize}
