% Acronyms
\newacronym[description={\glslink{itsg}{Issue Tracking System}}]
    {its}{ITS}{Issue Tracking System}

\newacronym[description={\glslink{awsg}{Amazon Web Services}}]
    {aws}{AWS}{Amazon Web Services}

\newacronym[description={\glslink{apig}{Application Program Interface}}]
    {api}{API}{Application Program Interface}

\newacronym{ceog}{CEO}{Chief Executive Officer}

\newacronym{hrg}{HR}{Human Resources}

\newacronym[description={\glslink{ci-cdg}{Continuous Integration \& Continuous Delivery}}]
    {ci-cd}{CI/CD}{Continuous Integration \& Continuous Delivery}

\newacronym[description={\glslink{rag-g}{Retrieval Augmented Generation}}]
    {rag}{RAG}{Retrieval Augmented Generation}

\newacronym[description={\glslink{llm-g}{Large Language Model}}]
    {llm}{LLM}{Large Language Model}

\newacronym[description={\glslink{knn-g}{K-Nearest Neighbors}}]
    {knn}{KNN}{K-Nearest Neighbors}

\newacronym{httpg}{HTTP}{Hypertext Transfer Protocol}

\newacronym{dtog}{DTO}{Data Transfer Object}

\newacronym{umlg}{UML}{Unified Modeling Language}

\newacronym{htmlg}{HTML}{Hypertext Markup Language}
\newacronym{cssg}{CSS}{Cascading Style Sheets}
% Glossary entries


\newglossaryentry{itsg} {
    name=\glslink{its}{ITS},
    text=Issue Tracking Sytem (ITS),
    sort=its,
    description={Con il termine \emph{Issue Tracking System (ITS) } ci si riferisce ad un sistema software utilizzato per registrare, monitorare e gestire \textit{ticket} che rappresentano \textit{task} da svolgere, problemi, richieste di miglioramento o altre attività da risolvere all’interno di un’organizzazione. Facilita la comunicazione tra i membri del team e migliora l’efficienza nella risoluzione dei problemi, consentendo di organizzare e prioritizzare il lavoro in modo efficace}
}

\newglossaryentry{awsg} {
    name=\glslink{aws}{AWS},
    text=Amazon Web Services (AWS),
    sort=aws,
    description={Con il termine \emph{AWS, Amazon Web Services} ci si riferisce ad una piattaforma di servizi cloud offerta da Amazon che offre un’ampia gamma di servizi di calcolo, archiviazione, database, analisi, applicazioni e distribuzione, permettendo alle aziende di scalare e crescere rapidamente senza dover gestire infrastrutture fisiche}
}

\newglossaryentry{User-stories} {
    name=User stories,
    text=User stories,
    sort=User-stories,
    description={Le \emph{User stories} rappresentano una pratica \textit{Agile}, che viene utilizzata per comprendere le esigenze dell'utente, mediante una descrizione informale, semplice e concisa della funzionalità}
}

\newglossaryentry{serverlessg}{
    name=Serverless,
    text=Serverless,
    sort=serverless,
    description={\emph{Serverless} è un modello di sviluppo di applicazioni \textit{cloud} in cui il fornitore del servizio (AWS nel contesto aziendale), gestisce automaticamente l'allocazione delle risorse richieste dall'applicazione. L'utente pagherà soltanto per il tempo in cui il codice è in esecuzione (\textit{pay as you go}), senza dover gestire l'infrastruttura sottostante}
}

\newglossaryentry{apig} {
    name=\glslink{api}{API},
    text=Application Program Interface,
    sort=api,
    description={In informatica, con il termine \emph{API, Application Programming Interface} si intende un insieme di protocolli che permette la comunicazione tra diverse applicazioni software. Funge da contratto di come una componente software deve interagire con un'altra, attraverso richieste e risposte}
}

\newglossaryentry{ci-cdg}{
    name=\glslink{ci-cd}{CI/CD},
    text=Continuous Integration \& Continuous Delivery,
    sort=ci-cd,
    description={In informatica, con il termine \emph{CI/CD, Continuous Integration \& Continuous Delivery} ci si riferisce ad una pratica di sviluppo in cui tutte le modifiche apportate al \textit{software} durante lo sviluppo, vengono integrate e testate automaticamente, in modo da garantire che il codice sia sempre funzionante e pronto per il rilascio}
}

\newglossaryentry{stageItg}{
    name=StageIT,
    text=StageIT,
    sort=stageIt,
    description={\emph{StageIT} è un evento promosso da Confindustria Veneto Est con la collaborazione dei dipartimenti di matematica e scienze statistiche dell'Università di Padova, per favorire l'incontro tra le aziende e gli studenti dei corsi di laurea di informatica, ingegneria informatica e statistica}
}

\newglossaryentry{rag-g}{
    name=\glslink{rag}{RAG},
    text=RAG,
    sort=rag,
    description={Con il termine \emph{RAG, Retrieval Augmented Generation} ci si riferisce ad una tecnica avanzata nell'ambito nell'elaborazione di linguaggio naturale, che combina il recupero di informazioni con la generazione di testo, per produrre risposte con la conoscenza del contesto, creando risposte coerenti e pertinenti alla domanda dell'utente}
}

\newglossaryentry{llm-g}{
    name=\glslink{llm}{LLM},
    text=LLM,
    sort=llm,
    description={Con il termine \emph{LLM, Large Language Model} ci si riferisce ad un modello di intelligenza artificiale progettato per generare testo in linguaggio naturale. Questi modelli sono addestrati su grandi quantità di dati testuali ed utilizzano tecniche di apprendimento automatico per generare risposte coerenti al contesto della domanda}
}

\newglossaryentry{embedding-g}{
    name=Embedding,
    text=embedding,
    sort=embedding,
    description={Un \emph{Embedding} è una rappresentazione numerica di oggetti, come parole o immagini in uno spazio vettoriale multidimensionale. Questa rappresentazione permette di catturare le relazioni semantiche tra gli oggetti, permettendo cosi di il confronto e l'analisi di similarità in modo più efficiente}
}

\newglossaryentry{knn-g}{
    name=\glslink{knn}{KNN},
    text=K-Nearest Neighbors,
    sort=knn,
    description={La tecnica \emph{K-Nearest Neighbors} è una tecnica utilizzata per la classificazione e la regressione che si basa sul concetto di vicinanza tra i dati. Viene etichettato un nuovo punto all'interno dello spazio vettoriale, in base all'etichetta dei K punti più vicini ad esso. Viene comunemente usata in ambito di \textit{machine learning} e sistemi di raccomandazione}
}

\newglossaryentry{promptenginnering-g}{
    name=Prompt Engineering,
    text=Prompt Engineering,
    sort=promptenginnering,
    description={Il \emph{Prompt Engineering} è un processo di progettazione e ottimizzazione dei \textit{prompt}, ovvero delle istruzioni o domande, utilizzate per guidare i modelli di intelligenza artificiale. L'obiettivo di tale processo è ottenere un \textit{prompt} che permetta di ottenere risposte più accurate, pertinenti e coerenti}
}

\newglossaryentry{uml-g} {
    name=UML,
    text=UML,
    sort=uml,
    description={Con il termine \emph{UML, Unified Modeling Language} ci si riferisce ad un linguaggio di modellazione visivo utilizzato per fonire una rappresentazione standard della progettazione di un sistema}
}

\newglossaryentry{iaGenerativag} {
    name=IA Generativa,
    text=IA Generativa,
    sort=iaGenerativa,
    description={Con il termine \emph{IA (Intelligenza Artificiale) Generativa} ci si riferisce ad un ramo dell'intelligenza artificiale che dato un contesto, è in grado di generare nuovi dati coerenti con esso. Questi modelli sono utilizzati per generare testo, immagini, musica e video}
}

