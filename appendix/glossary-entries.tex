% Acronyms
\newacronym[description={\glslink{itsg}{Issue Tracking System}}]
    {its}{ITS}{Issue Tracking System}

\newacronym[description={\glslink{awsg}{Amazon Web Services}}]
    {aws}{AWS}{Amazon Web Services}

\newacronym[description={\glslink{apig}{Application Program Interface}}]
    {api}{API}{Application Program Interface}

\newacronym{ceog}{CEO}{Chief Executive Officer}

\newacronym{hrg}{HR}{Human Resources}

\newacronym[description={\glslink{ci-cdg}{Continuous Integration \& Continuous Delivery}}]
    {ci-cd}{CI/CD}{Continuous Integration \& Continuous Delivery}

% Glossary entries

\newglossaryentry{itsg} {
    name=\glslink{its}{ITS},
    text=Issue Tracking Sytem (ITS),
    sort=its,
    description={Con il termine \emph{Issue Tracking System (ITS) } ci si riferisce ad un sistema software utilizzato per registrare, monitorare e gestire \textit{ticket} che rappresentano \textit{task} da svolgere, problemi, richieste di miglioramento o altre attività da risolvere all’interno di un’organizzazione. Facilita la comunicazione tra i membri del team e migliora l’efficienza nella risoluzione dei problemi, consentendo di organizzare e prioritizzare il lavoro in modo efficace}
}

\newglossaryentry{awsg} {
    name=\glslink{aws}{AWS},
    text=Amazon Web Services (AWS),
    sort=aws,
    description={Con il termine \emph{AWS, Amazon Web Services} ci si riferisce ad una piattaforma di servizi cloud offerta da Amazon che offre un’ampia gamma di servizi di calcolo, archiviazione, database, analisi, applicazioni e distribuzione, permettendo alle aziende di scalare e crescere rapidamente senza dover gestire infrastrutture fisiche}
}

\newglossaryentry{User-stories} {
    name=User stories,
    text=User stories,
    sort=User-stories,
    description={Le \emph{User stories} rappresentano una pratica \textit{Agile}, che viene utilizzata per comprendere le esigenze dell'utente, mediante una descrizione informale, semplice e concisa delle funzionalità}
}

\newglossaryentry{serverlessg}{
    name=Serverless,
    text=Serverless,
    sort=serverless,
    description={\emph{Serverless} è un modello di sviluppo di applicazioni \textit{cloud} in cui il fornitore del servizio (AWS nel contesto aziendale), gestisce automaticamente l'allocazione delle risorse richieste dall'applicazione. L'utente pagherà soltanto per il tempo in cui il codice è in esecuzione (\textit{pay as you go}), senza dover gestire l'infrastruttura sottostante}
}

\newglossaryentry{apig} {
    name=\glslink{api}{API},
    text=Application Program Interface,
    sort=api,
    description={In informatica, con il termine \emph{API, Application Program Interface} si intende un insieme di protocolli che permette la comunicazione tra diverse applicazioni software. Funge da contratto di come una componente software deve interagire con un'altra, attraverso richieste e risposte}
}

\newglossaryentry{ci-cdg}{
    name=\glslink{ci-cd}{CI/CD},
    text=Continuous Integration \& Continuous Delivery,
    sort=ci-cd,
    description={In informatica, con il termine \emph{CI/CD, Continuous Integration \& Continuous Delivery} ci si riferisce ad una pratica di sviluppo in cui tutte le modifiche apportate al \textit{software} durante lo sviluppo, vengono integrate e testate automaticamente, in modo da garantire che il codice sia sempre funzionante e pronto per il rilascio}
}
