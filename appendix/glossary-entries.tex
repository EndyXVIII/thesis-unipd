% Acronyms
\newacronym[description={\glslink{apig}{Application Program Interface}}]
    {api}{API}{Application Program Interface}

\newacronym[description={\glslink{umlg}{Unified Modeling Language}}]
    {uml}{UML}{Unified Modeling Language}

\newacronym[description={\glslink{itsg}{Issue Tracking System}}]
    {its}{ITS}{Issue Tracking System}

\newacronym[description={\glslink{awsg}{Amazon Web Services}}]
    {aws}{AWS}{Amazon Web Services}

\newacronym{ceog}{CEO}{Chief Executive Officer}

% Glossary entries
\newglossaryentry{apig} {
    name=\glslink{api}{API},
    text=Application Program Interface,
    sort=api,
    description={in informatica con il termine \emph{Application Programming Interface API} (ing. interfaccia di programmazione di un'applicazione) si indica ogni insieme di procedure disponibili al programmatore, di solito raggruppate a formare un set di strumenti specifici per l'espletamento di un determinato compito all'interno di un certo programma. La finalità è ottenere un'astrazione, di solito tra l'hardware e il programmatore o tra software a basso e quello ad alto livello semplificando così il lavoro di programmazione}
}

\newglossaryentry{umlg} {
    name=\glslink{uml}{UML},
    text=UML,
    sort=uml,
    description={in ingegneria del software \emph{UML, Unified Modeling Language} (ing. linguaggio di modellazione unificato) è un linguaggio di modellazione e specifica basato sul paradigma object-oriented. L'\emph{UML} svolge un'importantissima funzione di ``lingua franca'' nella comunità della progettazione e programmazione a oggetti. Gran parte della letteratura di settore usa tale linguaggio per descrivere soluzioni analitiche e progettuali in modo sintetico e comprensibile a un vasto pubblico}
}

\newglossaryentry{itsg} {
    name=\glslink{its}{ITS},
    text=ITS,
    sort=its,
    description={Con il termine \emph{Issue Tracking System (ITS) } si intende un sistema software utilizzato per registrare, monitorare e gestire ticket che rappresentano problemi, richieste di miglioramento o altre attività da risolvere all’interno di un’organizzazione. Facilita la comunicazione tra i membri del team e migliora l’efficienza nella risoluzione dei problemi, consentendo di organizzare e prioritizzare il lavoro in modo efficace. Utilizzato in supporto tecnico, sviluppo software e gestione dei progetti.}
}

\newglossaryentry{awsg} {
    name=\glslink{aws}{AWS},
    text=AWS,
    sort=aws,
    description={Con il termine \emph{AWS, Amazon Web Services} ci si riferisce ad una piattaforma di servizi cloud offerta da Amazon che offre un’ampia gamma di servizi di calcolo, archiviazione, database, analisi, applicazioni e distribuzione, permettendo alle aziende di scalare e crescere rapidamente senza dover gestire infrastrutture fisiche}
}

