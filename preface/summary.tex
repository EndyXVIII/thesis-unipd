\cleardoublepage
\phantomsection
\pdfbookmark{Sommario}{Sommario}
\begingroup
\let\clearpage\relax
\let\cleardoublepage\relax
\let\cleardoublepage\relax
\chapter*{Sommario}
Il presente documento descrive il lavoro svolto durante il periodo di \textit{stage} svolto presso l'azienda Zero12 s.r.l, avente sede a Padova, nel periodo compreso tra il 7 maggio e il 1 luglio 2024.
Sarà analizzata in primo luogo, la prima parte del progetto che ha portato alla creazione di un sistema di proposte di risoluzioni automatiche per i \textit{ticket} inseriti all'interno del sistema di \textit{ticketing} aziendale e successivamente
la seconda parte del progetto che ha portato alla creazione di un \textit{chatbot}, sempre per la generazione di proposte di risoluzioni automatiche, che nasce con l'obiettivo di essere un sistema aggiuntivo rispetto al sistema di proposte di risoluzioni automatiche già esistente.
Nella stesura del testo ho seguito alcune convenzioni tipografiche. Ho definito nel glossario, situato alla fine del documento, tutti i termini ambigui o di uso non comune che ho menzionato. Inoltre, ho evidenziato in corsivo i termini in lingua diversa dall’italiano.
La relazione è suddivisa in quattro capitoli principali, ognuno riguardante un aspetto dello \textit{stage} svolto: 
\begin{itemize}
    \item \textbf{Capitolo \ref{cap:introduzione}}: informazioni generali sull'azienda ospitante. Descrivo la loro organizzazione interna, la metodologia di lavoro e le tecnologie adottate. Presento il profilo della loro clientela e della loro propensione all'innovazione;
    \item \textbf{Capitolo \ref{cap:lo-stage}}: motivazioni della proposta dello \textit{stage} da parte dell'azienda. Analizzo le loro aspettative, quali vincoli ho dovuto rispettare e le mie motivazioni personali sulla scelta del progetto di \textit{stage};
    \item \textbf{Capitolo \ref{cap:descrizione-stage}}: descrivo cosa ho prodotto e come ho lavorato. Illustro le scelte progettuali e tecnologiche che ho adottato;
    \item \textbf{Capitolo \ref{cap:conclusioni}}: obiettivi raggiunti e difficoltà che ho incontrato. Valuto le competenze che ho acquisito e il divario formativo tra università e mondo del lavoro.
\end{itemize}


%\selectlanguage{english}
%\pdfbookmark{Abstract}{Abstract}
%\chapter*{Abstract}

%\selectlanguage{italian}

\endgroup

\vfill
