\cleardoublepage
\phantomsection
\pdfbookmark{Sommario}{Sommario}
\begingroup
\let\clearpage\relax
\let\cleardoublepage\relax
\let\cleardoublepage\relax

\chapter*{Sommario}

Il presente documento descrive il lavoro svolto durante il periodo di \textit{stage} svolto presso l'azienda Zero12 s.r.l, avente sede a Padova, nel periodo compreso tra il 7 maggio e il 1 luglio 2024.
Sarà analizzata in primo luogo, la prima parte del progetto che ha portato alla creazione di un sistema di proposte di risoluzioni automatiche per i \textit{ticket} inseriti all'interno del sistema di \textit{ticketing} aziendale e successivamente
la seconda parte del progetto che ha portato alla creazione di un \textit{chatbot}, sempre per la generazione di proposte di risoluzioni automatiche, che nasce con l'obiettivo di essere un sistema aggiuntivo rispetto al sistema di proposte di risoluzioni automatiche già esistente.
Riguardo alla stesura del testo, relativamente al documento ho adottato le seguenti convenzioni tipografiche:
\begin{itemize}
    \item I termini ambigui o di uso non comune menzionati, li ho definiti nel glossario, situato alla fine del documento;
    \item I termini in lingua diversa dall'italiano li ho evidenziati con il carattere \emph{corsivo}.
\end{itemize}

%\vfill
\pdfbookmark{Organizzazione del documento}{Organizzazione del documento}
\chapter*{Organizzazione del documento}
La relazione è suddivisa in quattro capitoli principali:
\begin{itemize}
    \item \textbf{Capitolo \ref{cap:introduzione}}: Descrizione del contesto aziendale in cui è stato inserito il mio \textit{stage};
    \item \textbf{Capitolo \ref{cap:lo-stage}}: Analisi della proposta di \textit{stage};
    \item \textbf{Capitolo \ref{cap:descrizione-stage}}: Svolgimento del progetto di \textit{stage} e descrizione delle attività svolte;
    \item \textbf{Capitolo \ref{cap:conclusioni}}: Valutazione retrospettiva dello \textit{stage}.
\end{itemize}


%\selectlanguage{english}
%\pdfbookmark{Abstract}{Abstract}
%\chapter*{Abstract}

%\selectlanguage{italian}

\endgroup

\vfill
